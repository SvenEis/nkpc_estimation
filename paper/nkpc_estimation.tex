\documentclass[11pt, a4paper, leqno]{article}
\usepackage{a4wide}
\usepackage[T1]{fontenc}
\usepackage[utf8]{inputenc}
\usepackage{float, afterpage, rotating, graphicx}
\usepackage{epstopdf}
\usepackage{longtable, booktabs, tabularx}
\usepackage{fancyvrb, moreverb, relsize}
\usepackage{eurosym, calc}
% \usepackage{chngcntr}
\usepackage{amsmath, amssymb, amsfonts, amsthm, bm}
\usepackage{caption}
\usepackage{mdwlist}
\usepackage{xfrac}
\usepackage{setspace}
\usepackage[dvipsnames]{xcolor}
\usepackage{subcaption}
\usepackage{minibox}
% \usepackage{pdf14} % Enable for Manuscriptcentral -- can't handle pdf 1.5
% \usepackage{endfloat} % Enable to move tables / figures to the end. Useful for some
% submissions.

%\usepackage[
%    natbib=true,
%    bibencoding=inputenc,
%    bibstyle=authoryear-ibid,
%    citestyle=authoryear-comp,
%    maxcitenames=3,
%    maxbibnames=10,
%    useprefix=false,
%    sortcites=true,
%    backend=biber
%]{biblatex}
%\AtBeginDocument{\toggletrue{blx@useprefix}}
%\AtBeginBibliography{\togglefalse{blx@useprefix}}
%\setlength{\bibitemsep}{1.5ex}
%\addbibresource{../paper/refs.bib}

\usepackage[unicode=true]{hyperref}
\hypersetup{
    colorlinks=true,
    linkcolor=black,
    anchorcolor=black,
    citecolor=NavyBlue,
    filecolor=black,
    menucolor=black,
    runcolor=black,
    urlcolor=NavyBlue
}


\widowpenalty=10000
\clubpenalty=10000

\setlength{\parskip}{1ex}
\setlength{\parindent}{0ex}
\setstretch{1.5}


\begin{document}

\title{The New Keynesian Phillips Curve: An Empirical Assessment\thanks{Sven Eis, University of Bonn. Email: \href{mailto:s6sveiss@uni-bonn.de}{\nolinkurl{s6sveiss [at] uni-bonn [dot] de}}.}}

\author{Sven Eis}

\date{
    {\bf Preliminary -- please do not quote}
    \\[1ex]
    March 31, 2023
}

\maketitle


\begin{abstract}
    I estimate the forward looking New Keynesian Phillips curve using measures of marignal cost as well as an ad-hoc output gap, using quarterly US data from $1960-2022$.
\end{abstract}

\clearpage


\section{Introduction} % (fold)
Over the past few decades, the New Keynesian Phillips Curve (NKPC) has become a cornerstone of macroeconomic models used by central banks and policymakers. The NKPC relates inflation to real marginal cost, which is typically modeled as a function of the output gap, defined as the deviation of actual output from natural level of output. Empirical estimates of the NKPC have been widely used to guide monetary policy, but the stability of the curve has been the subject of much debate in recent years.
This paper aims to estimate the NKPC for the US economy using quarterly data from 1960 to 2022. I build on the works of Gali and Coibion to provide a comprehensive analysis of the determinants of inflation dynamics in the US.
I use a similar econometric approach as Gali to estimate the NKPC, but I also incorporate the insights from Coibion to test whether the Phillips curve is still a reliable guide for monetary policy. Specifically, I investigate the impact of inflation expectations on the dynamics of inflation and test whether the Phillips curve has become flatter over time.

\textbf{Related Literature.} Gali introduced a structural econometric approach to estimate the NKPC, which has been widely adopted in the literature. Coibion and Gorodnichenko (2015) challenged the stability of the Phillips curve, arguing that it has become flatter over time due to the anchoring of inflation expectations.

The paper is organized as follows: the next chapter describes the theoretical background, chapter describes the econometric model, the estimation method and data, chapter presents the estimation results of the analysis, and discusses possible robustness checks, and chapter provides a conclusion and addresses potential further research. The project is build using the template from Gaudecker.


%section introduction (end)

\section{Theoretical Foundation}
The baseline model utilized in this research paper is the Representative Agent New Keynesian (RANK) model. This economic model consists of a continuum of monopolistic intermediate-goods firms, a perfectly competitive final-goods firm, a perfectly competitive representative household, and a central bank responsible for managing monetary policy. The upcoming sections of this paper will provide a detailed explanation of the objectives and constraints of the firm sector. For a complete understanding of the model equations, including the household problem and monetary policy rule, please refer to Appendix.
\subsection{Final-Goods Firm}
The final-goods firm is a representative, perfect competitive firm which uses intermediate goods as inputs for their production of the final good according to the following technology:
\begin{equation}
	Y_{t} = \left( \int_{0}^{1} Y_{t}(j)^{\frac{\epsilon -1}{\epsilon}} \,dj \right)^{\frac{\epsilon}{\epsilon -1}},
\end{equation}
where $Y_{t}(j)$ represents the intermediate good $j$, $Y_{t}$ the final good, and $\epsilon >1$ represents the constant price elasticity of demand.\\
The firm maximizes its profit subject to (s.t.) the production function, taking intermediate-goods prices and the final-goods price as given:
\begin{equation*}
	\begin{aligned}
		\max_{Y_{t}(j)} \quad & P_{t}Y_{t} = \int_{0}^{1} P_{t}(j)Y_{t}(j) \,dj\\
		\textrm{s.t.} \quad & Y_{t} = \left( \int_{0}^{1} Y_{t}(j)^{\frac{\epsilon -1}{\epsilon}} \,dj \right)^{\frac{\epsilon}{\epsilon -1}} \\
	\end{aligned}
\end{equation*}
After maximizing the latter problem, one find the input demand function:
\begin{equation}
		Y_{t}(j) = \left( \frac{P_{t}(j)}{P_{t}} \right)^{- \epsilon} Y_{t},
\end{equation}
where $P_{t}(j)$ is the price for the intermediate good $j$ and $P_{t}$ is the price for the final good.
As a reason of the zero profit condition, we get that:
\begin{equation}
		P_{t} = \left( \int_{0}^{1} P_{t}(j)^{1- \epsilon} \,dj \right)^{ \frac{1}{1- \epsilon}}
\end{equation}

\subsection{Intermediate-Goods Firms}
Within the model, there exists a continuous range of monopolistic firms, indexed by $j \in [0,1]$, each of which produce differentiated intermediate goods. These intermediate-goods firms are equipped with a production function that can be represented as follows:
\begin{equation}
		Y_{t}(j) = A_{t}N_{t}(j)^{1- \alpha},
\end{equation}
where $A_{t}$ denotes a common level of technology, $N_{t}(j)$ stands for labor, and $\alpha$ represents the associated elasticity.\\
Notably, wage levels are taken as given by the intermediate-goods firms, which are not able to freely adjust prices. Instead, following the work of Calvo, each firm has the ability to reset its price only with a probability of $1- \theta$ during a given period. Let $S(t) \in [0,1]$ denote the set of firms that do not re-optimize their posted prices in period $t$. The resulting aggregate price level can then be expressed as follows:
\begin{equation}
	\begin{aligned}
	P_{t} \quad & = \left[ \int_{S(t)}^{} P_{t-1}(i)^{1- \epsilon} di + (1- \theta) \left( P^{*}_{t} \right)^{1- \epsilon} \right]^{\frac{1}{1- \epsilon}}\\
		\quad & = \left[ \theta \left( P_{t-1} \right)^{1- \epsilon} + \left(1- \theta \right) \left( P^{*}_{t} \right)^{1- \epsilon} \right]^{\frac{1}{1- \epsilon}}\\
	\end{aligned}
\end{equation}
As $P_{t}$ may be non-stationary, and in order to perform a log-linearization around a stationary point, it is necessary to raise both sides of the equation to the power of $1- \epsilon$ and divide both sides by $P_{t-1}$. This allows me to continue my analysis using inflation, $\pi_{t}$, as follows:
\begin{equation}
	\pi_{t}^{1- \epsilon} = \theta + (1- \theta) \left( \frac{P^{*}_{t}}{P_{t-1}} \right)^{1- \epsilon}
\end{equation}
Performing a log-linearization of equation around zero results in the following expression:
\begin{equation}
	p_{t} = \theta p_{t-1} \left( 1- \theta \right) p^{*}_{t}
\end{equation}
The dynamic optimization problem faced by the firm for determining the optimal price setting can be expressed as follows:
\begin{equation}
	\begin{aligned}
	\max_{P^{*}_{t}}  \quad & \sum_{K=0}^{\infty} \theta^{K} E_{t} \left\{ \frac{\Lambda_{t,t+K}}{P_{t+K}} \left[ P_{t}^{*} Y_{t+K \mid t} - \Psi \left( Y_{t+K \mid t} \right) \right] \right\} \\
	\textrm{s.t.} \quad & Y_{t+K \mid t} = \left( \frac{P^{*}_{t}}{P_{t+K}} \right) Y_{t+K}, \\
	\end{aligned}
\end{equation}
where $\Lambda_{t,t+K}$ is the stochastic discount factor, and $\Psi \left( Y_{t+K \mid t} \right)$ is the nominal cost.\\
By taking the first-order condition (FOC) and subsequently performing a log-linearization, we arrive at the following expression:
\begin{equation}
	p^{*}_{t} = \mu + \left(1-\beta \theta \right) \sum_{K=0}^{\infty} \left(\theta \beta \right)^{K} E_{t} \psi_{t+K \mid t},
\end{equation}
where $\psi_{t+K \mid t}=\Psi^{'}_{t+K \mid t}$ is the nominal marginal cost.
\subsection{The New Keynesian Phillips Curve}
In order to derive the New Keynesian Phillips Curve (NKPC), it is necessary to rearrange equation. A thorough explanation of the derivation can be found in Appendix. The resulting expression for the NKPC, in terms of the markup gap, is as follows:
\begin{equation}
	\pi_{t} = \beta E_{t} \left\{ \pi_{t+1} \right\} - \lambda \hat{\mu}_{t},
\end{equation}
where $\hat{\mu}_{t}= \mu_{t}-\mu$ represents the deviation of the markup from its steady state level. By utilizing the log-linearized form of the average real marginal cost, which depends on the real wage and the marginal product of labor and is expressed as $mc_{t} = w_{t} - p_{t} - mpn_{t} $, the NKPC can be alternatively presented in terms of the output gap. This results in the following equation, where the logarithmic deviation of the real marginal cost from its steady state level is proportionate to the logarithmic deviation of the output from its flexible price equivalent:
\begin{equation}
	\hat{\mu}_{t}= \left[ \sigma + \frac{\phi + \alpha}{1- \alpha} \right] \left( y_{t} - y_{t}^{n} \right)
\end{equation}
Combining this with equation results in the following expression for the NKPC in terms of the output gap:
\begin{equation}
	\pi_{t} = \beta E_{t} \left\{ \pi_{t+1} \right\} + \kappa \widetilde{y}_{t},
\end{equation}
where $\kappa \equiv \lambda \left[ \sigma + \frac{\phi + \alpha}{1- \alpha} \right] $ is the slope of the NKPC.

\section{Estimation and Data}
The estimation of the NKPC can be achieved in several ways. By looking at the past literature, estimation methods as well as data sources varied a lot. In order to have a complete analysis, I will conduct all the different tools and data which were used before.
\subsection{Estimation}
The first estimation method, which will be used as a baseline in this paper, is the ordinary least square (OLS) regression.
\subsubsection{Coibion and Gorodnichenko (2015)}
OLS Regression

\subsubsection{Gali and Gertler (1999)}

\subsection{Data}
For the estimation I use quarterly US data from $1960Q1-2022Q4$.

\section{Results and Robustness Checks}
\subsection{Results}

\begin{figure}[ht!]
	\caption{Estimation results Backward Expectations.}
	\centering
	\resizebox{\textwidth}{!}{
	\begin{minipage}{.49\textwidth}
		\begin{subfigure}{\textwidth}
		\centering
			\includegraphics[width=\textwidth]{../bld/python/figures/GDP_BackExp_OLS.pdf}
           	\end{subfigure}\\
	 	\begin{subfigure}{\textwidth}
            	\centering
           		\includegraphics[width=\textwidth]{../bld/python/figures/Labor_share_BackExp_OLS.pdf}
            	\end{subfigure}\\
	\end{minipage}
	%\hfill
        \begin{minipage}{.49\textwidth}
        	\begin{subfigure}{\textwidth}
                    \centering
                       \includegraphics[width=\textwidth]{../bld/python/figures/Unemp_BackExp_OLS.pdf}
                    \end{subfigure}\\
            \begin{subfigure}{\textwidth}
                    \centering
                    \includegraphics[width=\textwidth]{../bld/python/figures/Unemp_Gap_BackExp_OLS.pdf}
                    \end{subfigure}\\
        \end{minipage}
        \end{minipage}}
    	\bigskip
	\begin{minipage}{\textwidth}%
		\footnotesize\setlength{\baselineskip}{11pt}%
		\bigskip \textit{Notes:} The above figures show the  to a risk premium shock for specific variables. The ordinate shows the variable's proportional deviation from its steady state value in percent. The abscissa shows the quarters from zero to 40. The red line shows the IRF to a risk premium shock in the model with price level targeting and the blue line shows the IRF in a model with inflation level targeting.
	\end{minipage}
\end{figure}

\begin{figure}[ht!]
	\caption{Estimation results MSC.}
	\centering
	\resizebox{\textwidth}{!}{
	\begin{minipage}{.49\textwidth}
		\begin{subfigure}{\textwidth}
		\centering
			\includegraphics[width=\textwidth]{../bld/python/figures/GDP_MSC_OLS.pdf}
           	\end{subfigure}\\
	 	\begin{subfigure}{\textwidth}
            	\centering
           		\includegraphics[width=\textwidth]{../bld/python/figures/Labor_share_MSC_OLS.pdf}
            	\end{subfigure}\\
	\end{minipage}
	%\hfill
        \begin{minipage}{.49\textwidth}
        	\begin{subfigure}{\textwidth}
                    \centering
                       \includegraphics[width=\textwidth]{../bld/python/figures/Unemp_MSC_OLS.pdf}
                    \end{subfigure}\\
            \begin{subfigure}{\textwidth}
                    \centering
                    \includegraphics[width=\textwidth]{../bld/python/figures/Unemp_Gap_MSC_OLS.pdf}
                    \end{subfigure}\\
        \end{minipage}
        \end{minipage}}
    	\bigskip
	\begin{minipage}{\textwidth}%
		\footnotesize\setlength{\baselineskip}{11pt}%
		\bigskip \textit{Notes:} The above figures show the  to a risk premium shock for specific variables. The ordinate shows the variable's proportional deviation from its steady state value in percent. The abscissa shows the quarters from zero to 40. The red line shows the IRF to a risk premium shock in the model with price level targeting and the blue line shows the IRF in a model with inflation level targeting.
	\end{minipage}
\end{figure}


\input{../bld/python/tables/GDP_BackExp_OLS.tex}
\input{../bld/python/tables/Labor_share_BackExp_OLS.tex}
\input{../bld/python/tables/Unemp_BackExp_OLS.tex}
\input{../bld/python/tables/Unemp_Gap_BackExp_OLS.tex}
\clearpage
\input{../bld/python/tables/GDP_MSC_OLS.tex}
\input{../bld/python/tables/Labor_share_MSC_OLS.tex}
\input{../bld/python/tables/Unemp_MSC_OLS.tex}
\input{../bld/python/tables/Unemp_Gap_MSC_OLS.tex}


\subsection{Robustness Checks}


\section{Conclusion}

\clearpage

%\setstretch{1}
%\printbibliography
%\setstretch{1.5}

\clearpage

\section*{\Huge Appendix}
\appendix
\setcounter{equation}{0}
\numberwithin{equation}{section}
\section{RANK model}
\subsection{Households}
In the model, I posit the existence of a continuum of a representitive perfectly competitive household maximising its expected lifetime utility $U \left( C_{t},N_{t} \right)$ at period $t = 0$. Consumers minimise expenditure given the consumption level of composite good $C_{t}$. Moreover, I assume a standard constant relative risk aversion (CRRA) functional form of additively separable consumption and labor:
\begin{equation}
    Unemp
\end{equation}

\subsection{Firms}

\subsection{Monetary Policy}

\subsection{Market Clearing}

\section{Sensitivity Analysis}
\subsection{Plots}

\begin{figure}[ht!]
	\caption{Breakpoint analysis GDP - Backward Expectations.}
	\centering
	\resizebox{\textwidth}{!}{
	\begin{minipage}{.3\textwidth}
		\begin{subfigure}{\textwidth}
		\centering
			\includegraphics[width=\textwidth]{../bld/python/figures/GDP_BackExp_OLS_1961-04-01.pdf}
           	\end{subfigure}\\
	 	\begin{subfigure}{\textwidth}
            	\centering
           		\includegraphics[width=\textwidth]{../bld/python/figures/GDP_BackExp_OLS_1984-10-01.pdf}
            	\end{subfigure}\\
	\end{minipage}
	%\hfill
        \begin{minipage}{.3\textwidth}
        	\begin{subfigure}{\textwidth}
                    \centering
                       \includegraphics[width=\textwidth]{../bld/python/figures/GDP_BackExp_OLS_2007-07-01.pdf}
                    \end{subfigure}\\
            \begin{subfigure}{\textwidth}
                    \centering
                    \includegraphics[width=\textwidth]{../bld/python/figures/GDP_BackExp_OLS_2013-01-01.pdf}
                    \end{subfigure}\\
        \end{minipage}
       % \hfill
        \begin{minipage}{.3\textwidth}
            \begin{subfigure}{\textwidth}
                \centering
                \includegraphics[width=\textwidth]{../bld/python/figures/GDP_BackExp_OLS_2020-04-01.pdf}
                \end{subfigure}\\
        \end{minipage}}
    	\bigskip
	\begin{minipage}{\textwidth}%
		\footnotesize\setlength{\baselineskip}{11pt}%
		\bigskip \textit{Notes:} The above figures show the  to a risk premium shock for specific variables. The ordinate shows the variable's proportional deviation from its steady state value in percent. The abscissa shows the quarters from zero to 40. The red line shows the IRF to a risk premium shock in the model with price level targeting and the blue line shows the IRF in a model with inflation level targeting.
	\end{minipage}
\end{figure}

\begin{figure}[ht!]
	\caption{Breakpoint analysis Labor share - Backward Expectations.}
	\centering
	\resizebox{\textwidth}{!}{
	\begin{minipage}{.3\textwidth}
		\begin{subfigure}{\textwidth}
		\centering
			\includegraphics[width=\textwidth]{../bld/python/figures/Labor_share_BackExp_OLS_1961-04-01.pdf}
           	\end{subfigure}\\
	 	\begin{subfigure}{\textwidth}
            	\centering
           		\includegraphics[width=\textwidth]{../bld/python/figures/Labor_share_BackExp_OLS_1984-10-01.pdf}
            	\end{subfigure}\\
	\end{minipage}
	%\hfill
        \begin{minipage}{.3\textwidth}
        	\begin{subfigure}{\textwidth}
                    \centering
                       \includegraphics[width=\textwidth]{../bld/python/figures/Labor_share_BackExp_OLS_2007-07-01.pdf}
                    \end{subfigure}\\
            \begin{subfigure}{\textwidth}
                    \centering
                    \includegraphics[width=\textwidth]{../bld/python/figures/Labor_share_BackExp_OLS_2013-01-01.pdf}
                    \end{subfigure}\\
        \end{minipage}
       % \hfill
        \begin{minipage}{.3\textwidth}
            \begin{subfigure}{\textwidth}
                \centering
                \includegraphics[width=\textwidth]{../bld/python/figures/Labor_share_BackExp_OLS_2020-04-01.pdf}
                \end{subfigure}\\
        \end{minipage}}
    	\bigskip
	\begin{minipage}{\textwidth}%
		\footnotesize\setlength{\baselineskip}{11pt}%
		\bigskip \textit{Notes:} The above figures show the  to a risk premium shock for specific variables. The ordinate shows the variable's proportional deviation from its steady state value in percent. The abscissa shows the quarters from zero to 40. The red line shows the IRF to a risk premium shock in the model with price level targeting and the blue line shows the IRF in a model with inflation level targeting.
	\end{minipage}
\end{figure}

\begin{figure}[ht!]
	\caption{Breakpoint analysis Unemployment - Backward Expectation.}
	\centering
	\resizebox{\textwidth}{!}{
	\begin{minipage}{.3\textwidth}
		\begin{subfigure}{\textwidth}
		\centering
			\includegraphics[width=\textwidth]{../bld/python/figures/Unemp_BackExp_OLS_1961-04-01.pdf}
           	\end{subfigure}\\
	 	\begin{subfigure}{\textwidth}
            	\centering
           		\includegraphics[width=\textwidth]{../bld/python/figures/Unemp_BackExp_OLS_1984-10-01.pdf}
            	\end{subfigure}\\
	\end{minipage}
	%\hfill
        \begin{minipage}{.3\textwidth}
        	\begin{subfigure}{\textwidth}
                    \centering
                       \includegraphics[width=\textwidth]{../bld/python/figures/Unemp_BackExp_OLS_2007-07-01.pdf}
                    \end{subfigure}\\
            \begin{subfigure}{\textwidth}
                    \centering
                    \includegraphics[width=\textwidth]{../bld/python/figures/Unemp_BackExp_OLS_2013-01-01.pdf}
                    \end{subfigure}\\
        \end{minipage}
       % \hfill
        \begin{minipage}{.3\textwidth}
            \begin{subfigure}{\textwidth}
                \centering
                \includegraphics[width=\textwidth]{../bld/python/figures/Unemp_BackExp_OLS_2020-04-01.pdf}
                \end{subfigure}\\
        \end{minipage}}
    	\bigskip
	\begin{minipage}{\textwidth}%
		\footnotesize\setlength{\baselineskip}{11pt}%
		\bigskip \textit{Notes:} The above figures show the  to a risk premium shock for specific variables. The ordinate shows the variable's proportional deviation from its steady state value in percent. The abscissa shows the quarters from zero to 40. The red line shows the IRF to a risk premium shock in the model with price level targeting and the blue line shows the IRF in a model with inflation level targeting.
	\end{minipage}
\end{figure}

\begin{figure}[ht!]
	\caption{Breakpoint analysis Unemployment Gap - Backward Expectation.}
	\centering
	\resizebox{\textwidth}{!}{
	\begin{minipage}{.3\textwidth}
		\begin{subfigure}{\textwidth}
		\centering
			\includegraphics[width=\textwidth]{../bld/python/figures/Unemp_Gap_BackExp_OLS_1961-04-01.pdf}
           	\end{subfigure}\\
	 	\begin{subfigure}{\textwidth}
            	\centering
           		\includegraphics[width=\textwidth]{../bld/python/figures/Unemp_Gap_BackExp_OLS_1984-10-01.pdf}
            	\end{subfigure}\\
	\end{minipage}
	%\hfill
        \begin{minipage}{.3\textwidth}
        	\begin{subfigure}{\textwidth}
                    \centering
                       \includegraphics[width=\textwidth]{../bld/python/figures/Unemp_Gap_BackExp_OLS_2007-07-01.pdf}
                    \end{subfigure}\\
            \begin{subfigure}{\textwidth}
                    \centering
                    \includegraphics[width=\textwidth]{../bld/python/figures/Unemp_Gap_BackExp_OLS_2013-01-01.pdf}
                    \end{subfigure}\\
        \end{minipage}
       % \hfill
        \begin{minipage}{.3\textwidth}
            \begin{subfigure}{\textwidth}
                \centering
                \includegraphics[width=\textwidth]{../bld/python/figures/Unemp_Gap_BackExp_OLS_2020-04-01.pdf}
                \end{subfigure}\\
        \end{minipage}}
    	\bigskip
	\begin{minipage}{\textwidth}%
		\footnotesize\setlength{\baselineskip}{11pt}%
		\bigskip \textit{Notes:} The above figures show the  to a risk premium shock for specific variables. The ordinate shows the variable's proportional deviation from its steady state value in percent. The abscissa shows the quarters from zero to 40. The red line shows the IRF to a risk premium shock in the model with price level targeting and the blue line shows the IRF in a model with inflation level targeting.
	\end{minipage}
\end{figure}

\begin{figure}[ht!]
	\caption{Breakpoint analysis GDP - MSC.}
	\centering
	\resizebox{\textwidth}{!}{
	\begin{minipage}{.3\textwidth}
		\begin{subfigure}{\textwidth}
		\centering
			\includegraphics[width=\textwidth]{../bld/python/figures/GDP_MSC_OLS_1961-04-01.pdf}
           	\end{subfigure}\\
	 	\begin{subfigure}{\textwidth}
            	\centering
           		\includegraphics[width=\textwidth]{../bld/python/figures/GDP_MSC_OLS_1984-10-01.pdf}
            	\end{subfigure}\\
	\end{minipage}
	%\hfill
        \begin{minipage}{.3\textwidth}
        	\begin{subfigure}{\textwidth}
                    \centering
                       \includegraphics[width=\textwidth]{../bld/python/figures/GDP_MSC_OLS_2007-07-01.pdf}
                    \end{subfigure}\\
            \begin{subfigure}{\textwidth}
                    \centering
                    \includegraphics[width=\textwidth]{../bld/python/figures/GDP_MSC_OLS_2013-01-01.pdf}
                    \end{subfigure}\\
        \end{minipage}
       % \hfill
        \begin{minipage}{.3\textwidth}
            \begin{subfigure}{\textwidth}
                \centering
                \includegraphics[width=\textwidth]{../bld/python/figures/GDP_MSC_OLS_2020-04-01.pdf}
                \end{subfigure}\\
        \end{minipage}}
    	\bigskip
	\begin{minipage}{\textwidth}%
		\footnotesize\setlength{\baselineskip}{11pt}%
		\bigskip \textit{Notes:} The above figures show the  to a risk premium shock for specific variables. The ordinate shows the variable's proportional deviation from its steady state value in percent. The abscissa shows the quarters from zero to 40. The red line shows the IRF to a risk premium shock in the model with price level targeting and the blue line shows the IRF in a model with inflation level targeting.
	\end{minipage}
\end{figure}

\begin{figure}[ht!]
	\caption{Breakpoint analysis Labor share - MSC.}
	\centering
	\resizebox{\textwidth}{!}{
	\begin{minipage}{.3\textwidth}
		\begin{subfigure}{\textwidth}
		\centering
			\includegraphics[width=\textwidth]{../bld/python/figures/Labor_share_MSC_OLS_1961-04-01.pdf}
           	\end{subfigure}\\
	 	\begin{subfigure}{\textwidth}
            	\centering
           		\includegraphics[width=\textwidth]{../bld/python/figures/Labor_share_MSC_OLS_1984-10-01.pdf}
            	\end{subfigure}\\
	\end{minipage}
	%\hfill
        \begin{minipage}{.3\textwidth}
        	\begin{subfigure}{\textwidth}
                    \centering
                       \includegraphics[width=\textwidth]{../bld/python/figures/Labor_share_MSC_OLS_2007-07-01.pdf}
                    \end{subfigure}\\
            \begin{subfigure}{\textwidth}
                    \centering
                    \includegraphics[width=\textwidth]{../bld/python/figures/Labor_share_MSC_OLS_2013-01-01.pdf}
                    \end{subfigure}\\
        \end{minipage}
       % \hfill
        \begin{minipage}{.3\textwidth}
            \begin{subfigure}{\textwidth}
                \centering
                \includegraphics[width=\textwidth]{../bld/python/figures/Labor_share_MSC_OLS_2020-04-01.pdf}
                \end{subfigure}\\
        \end{minipage}}
    	\bigskip
	\begin{minipage}{\textwidth}%
		\footnotesize\setlength{\baselineskip}{11pt}%
		\bigskip \textit{Notes:} The above figures show the  to a risk premium shock for specific variables. The ordinate shows the variable's proportional deviation from its steady state value in percent. The abscissa shows the quarters from zero to 40. The red line shows the IRF to a risk premium shock in the model with price level targeting and the blue line shows the IRF in a model with inflation level targeting.
	\end{minipage}
\end{figure}

\begin{figure}[ht!]
	\caption{Breakpoint analysis Unemployment - MSC.}
	\centering
	\resizebox{\textwidth}{!}{
	\begin{minipage}{.3\textwidth}
		\begin{subfigure}{\textwidth}
		\centering
			\includegraphics[width=\textwidth]{../bld/python/figures/Unemp_MSC_OLS_1961-04-01.pdf}
           	\end{subfigure}\\
	 	\begin{subfigure}{\textwidth}
            	\centering
           		\includegraphics[width=\textwidth]{../bld/python/figures/Unemp_MSC_OLS_1984-10-01.pdf}
            	\end{subfigure}\\
	\end{minipage}
	%\hfill
        \begin{minipage}{.3\textwidth}
        	\begin{subfigure}{\textwidth}
                    \centering
                       \includegraphics[width=\textwidth]{../bld/python/figures/Unemp_MSC_OLS_2007-07-01.pdf}
                    \end{subfigure}\\
            \begin{subfigure}{\textwidth}
                    \centering
                    \includegraphics[width=\textwidth]{../bld/python/figures/Unemp_MSC_OLS_2013-01-01.pdf}
                    \end{subfigure}\\
        \end{minipage}
       % \hfill
        \begin{minipage}{.3\textwidth}
            \begin{subfigure}{\textwidth}
                \centering
                \includegraphics[width=\textwidth]{../bld/python/figures/Unemp_MSC_OLS_2020-04-01.pdf}
                \end{subfigure}\\
        \end{minipage}}
    	\bigskip
	\begin{minipage}{\textwidth}%
		\footnotesize\setlength{\baselineskip}{11pt}%
		\bigskip \textit{Notes:} The above figures show the  to a risk premium shock for specific variables. The ordinate shows the variable's proportional deviation from its steady state value in percent. The abscissa shows the quarters from zero to 40. The red line shows the IRF to a risk premium shock in the model with price level targeting and the blue line shows the IRF in a model with inflation level targeting.
	\end{minipage}
\end{figure}

\begin{figure}[ht!]
	\caption{Breakpoint analysis Unemployment Gap - MSC.}
	\centering
	\resizebox{\textwidth}{!}{
	\begin{minipage}{.3\textwidth}
		\begin{subfigure}{\textwidth}
		\centering
			\includegraphics[width=\textwidth]{../bld/python/figures/Unemp_Gap_MSC_OLS_1961-04-01.pdf}
           	\end{subfigure}\\
	 	\begin{subfigure}{\textwidth}
            	\centering
           		\includegraphics[width=\textwidth]{../bld/python/figures/Unemp_Gap_MSC_OLS_1984-10-01.pdf}
            	\end{subfigure}\\
	\end{minipage}
	%\hfill
        \begin{minipage}{.3\textwidth}
        	\begin{subfigure}{\textwidth}
                    \centering
                       \includegraphics[width=\textwidth]{../bld/python/figures/Unemp_Gap_MSC_OLS_2007-07-01.pdf}
                    \end{subfigure}\\
            \begin{subfigure}{\textwidth}
                    \centering
                    \includegraphics[width=\textwidth]{../bld/python/figures/Unemp_Gap_MSC_OLS_2013-01-01.pdf}
                    \end{subfigure}\\
        \end{minipage}
       % \hfill
        \begin{minipage}{.3\textwidth}
            \begin{subfigure}{\textwidth}
                \centering
                \includegraphics[width=\textwidth]{../bld/python/figures/Unemp_Gap_MSC_OLS_2020-04-01.pdf}
                \end{subfigure}\\
        \end{minipage}}
    	\bigskip
	\begin{minipage}{\textwidth}%
		\footnotesize\setlength{\baselineskip}{11pt}%
		\bigskip \textit{Notes:} The above figures show the  to a risk premium shock for specific variables. The ordinate shows the variable's proportional deviation from its steady state value in percent. The abscissa shows the quarters from zero to 40. The red line shows the IRF to a risk premium shock in the model with price level targeting and the blue line shows the IRF in a model with inflation level targeting.
	\end{minipage}
\end{figure}

% The chngctr package is needed for the following lines.
% \counterwithin{table}{section}
% \counterwithin{figure}{section}
\clearpage
\subsection{Tables}
\input{../bld/python/tables/GDP_BackExp_OLS_1961-04-01.tex}
\input{../bld/python/tables/GDP_BackExp_OLS_1984-10-01.tex}
\input{../bld/python/tables/GDP_BackExp_OLS_2007-07-01.tex}
\input{../bld/python/tables/GDP_BackExp_OLS_2013-01-01.tex}
\input{../bld/python/tables/GDP_BackExp_OLS_2020-04-01.tex}
\clearpage
\input{../bld/python/tables/Labor_share_BackExp_OLS_1961-04-01.tex}
\input{../bld/python/tables/Labor_share_BackExp_OLS_1984-10-01.tex}
\input{../bld/python/tables/Labor_share_BackExp_OLS_2007-07-01.tex}
\input{../bld/python/tables/Labor_share_BackExp_OLS_2013-01-01.tex}
\input{../bld/python/tables/Labor_share_BackExp_OLS_2020-04-01.tex}
\clearpage
\input{../bld/python/tables/Unemp_BackExp_OLS_1961-04-01.tex}
\input{../bld/python/tables/Unemp_BackExp_OLS_1984-10-01.tex}
\input{../bld/python/tables/Unemp_BackExp_OLS_2007-07-01.tex}
\input{../bld/python/tables/Unemp_BackExp_OLS_2013-01-01.tex}
\input{../bld/python/tables/Unemp_BackExp_OLS_2020-04-01.tex}
\clearpage
\input{../bld/python/tables/Unemp_Gap_BackExp_OLS_1961-04-01.tex}
\input{../bld/python/tables/Unemp_Gap_BackExp_OLS_1984-10-01.tex}
\input{../bld/python/tables/Unemp_Gap_BackExp_OLS_2007-07-01.tex}
\input{../bld/python/tables/Unemp_Gap_BackExp_OLS_2013-01-01.tex}
\input{../bld/python/tables/Unemp_Gap_BackExp_OLS_2020-04-01.tex}
\clearpage
\input{../bld/python/tables/GDP_MSC_OLS_1961-04-01.tex}
\input{../bld/python/tables/GDP_MSC_OLS_1984-10-01.tex}
\input{../bld/python/tables/GDP_MSC_OLS_2007-07-01.tex}
\input{../bld/python/tables/GDP_MSC_OLS_2013-01-01.tex}
\input{../bld/python/tables/GDP_MSC_OLS_2020-04-01.tex}
\clearpage
\input{../bld/python/tables/Labor_share_MSC_OLS_1961-04-01.tex}
\input{../bld/python/tables/Labor_share_MSC_OLS_1984-10-01.tex}
\input{../bld/python/tables/Labor_share_MSC_OLS_2007-07-01.tex}
\input{../bld/python/tables/Labor_share_MSC_OLS_2013-01-01.tex}
\input{../bld/python/tables/Labor_share_MSC_OLS_2020-04-01.tex}
\clearpage
\input{../bld/python/tables/Unemp_MSC_OLS_1961-04-01.tex}
\input{../bld/python/tables/Unemp_MSC_OLS_1984-10-01.tex}
\input{../bld/python/tables/Unemp_MSC_OLS_2007-07-01.tex}
\input{../bld/python/tables/Unemp_MSC_OLS_2013-01-01.tex}
\input{../bld/python/tables/Unemp_MSC_OLS_2020-04-01.tex}
\clearpage
\input{../bld/python/tables/Unemp_Gap_MSC_OLS_1961-04-01.tex}
\input{../bld/python/tables/Unemp_Gap_MSC_OLS_1984-10-01.tex}
\input{../bld/python/tables/Unemp_Gap_MSC_OLS_2007-07-01.tex}
\input{../bld/python/tables/Unemp_Gap_MSC_OLS_2013-01-01.tex}
\input{../bld/python/tables/Unemp_Gap_MSC_OLS_2020-04-01.tex}
\end{document}
