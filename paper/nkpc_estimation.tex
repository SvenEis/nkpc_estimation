\documentclass[11pt, a4paper, leqno]{article}
\usepackage{a4wide}
\usepackage[T1]{fontenc}
\usepackage[utf8]{inputenc}
\usepackage{float, afterpage, rotating, graphicx}
\usepackage{epstopdf}
\usepackage{longtable, booktabs, tabularx}
\usepackage{fancyvrb, moreverb, relsize}
\usepackage{eurosym, calc}
% \usepackage{chngcntr}
\usepackage{amsmath, amssymb, amsfonts, amsthm, bm}
\usepackage{caption}
\usepackage{mdwlist}
\usepackage{xfrac}
\usepackage{setspace}
\usepackage[dvipsnames]{xcolor}
\usepackage{subcaption}
\usepackage{minibox}
% \usepackage{pdf14} % Enable for Manuscriptcentral -- can't handle pdf 1.5
% \usepackage{endfloat} % Enable to move tables / figures to the end. Useful for some
% submissions.

\usepackage[
    natbib=true,
    bibencoding=inputenc,
    bibstyle=authoryear-ibid,
    citestyle=authoryear-comp,
    maxcitenames=3,
    maxbibnames=10,
    useprefix=false,
    sortcites=true,
    backend=biber
]{biblatex}
\AtBeginDocument{\toggletrue{blx@useprefix}}
\AtBeginBibliography{\togglefalse{blx@useprefix}}
\setlength{\bibitemsep}{1.5ex}
\addbibresource{../../paper/refs.bib}

\usepackage[unicode=true]{hyperref}
\hypersetup{
    colorlinks=true,
    linkcolor=black,
    anchorcolor=black,
    citecolor=NavyBlue,
    filecolor=black,
    menucolor=black,
    runcolor=black,
    urlcolor=NavyBlue
}


\widowpenalty=10000
\clubpenalty=10000

\setlength{\parskip}{1ex}
\setlength{\parindent}{0ex}
\setstretch{1.5}


\begin{document}

\title{The New Keynesian Phillips Curve: An Empirical Assessment\thanks{Sven Eis, University of Bonn. Email: \href{mailto:s6sveiss@uni-bonn.de}{\nolinkurl{s6sveiss [at] uni-bonn [dot] de}}.}}

\author{Sven Eis}

\date{
    {\bf Preliminary -- please do not quote}
    \\[1ex]
    March 31, 2023
}

\maketitle


\begin{abstract}
    I estimate the forward looking New Keynesian Phillips curve using measures of marignal cost as well as an ad-hoc output gap, using quarterly US data from $1960-2022$.
\end{abstract}

\clearpage


\section{Introduction} % (fold)
\label{sec:introduction}

The project is build using the template from \citet{GaudeckerEconProjectTemplates}.

\textbf{Related Literature.} xxxx\\
\\
The paper is organized as follows: the next chapter \ref{theory} describes the theoretical background, chapter \ref{estimation} describes the econometric model, the estimation method and data, chapter \ref{results} presents the estimation results of the analysis, and discusses possible robustness checks, and chapter \ref{conclusion} provides a conclusion and addresses potential further research.


%section introduction (end)

\section{Theoretical Foundation} \label{theory}
This research paper employs a basic model known as the Representative Agent New Keynesian (RANK) model as the baseline. The economy depicted in this model comprises a continuum of monopolistic intermediate-goods firms, a perfectly competitive final-goods firm, a perfectly competitive representative household, and a central bank that manages monetary policy. The forthcoming sections will elaborate on the objectives and constraints of the firm sector. For a comprehensive understanding of the model equations, including the household problem and monetary policy rule, please refer to Appendix \ref{sec:app:model}.

\subsection{Firms}
\subsubsection{Final-Goods Firm}
The final-goods firm is a representative, perfect competitive firm which uses intermediate goods as inputs for their production of the following good according to the following technology:
\begin{equation}
	Y_{t} = \left( \int_{0}^{1} Y_{t}(j)^{\frac{\epsilon -1}{\epsilon}} \,dj \right)^{\frac{\epsilon}{\epsilon -1}},
\end{equation}
where $Y_{t}(j)$ represents the intermediate good $j$, $Y_{t}$ the final good, and $\epsilon >1$ represents the constant price elasticity of demand.\\
The firm maximizes its profits subject to (s.t.) the production function, taking intermediate-goods prices and the final-good price as given:
\begin{equation*}
	\begin{aligned}
		\max_{Y_{t}(j)} \quad & P_{t}Y_{t} = \int_{0}^{1} P_{t}(j)Y_{t}(j) \,dj\\
		\textrm{s.t.} \quad & Y_{t} = \left( \int_{0}^{1} Y_{t}(j)^{\frac{\epsilon -1}{\epsilon}} \,dj \right)^{\frac{\epsilon}{\epsilon -1}} \\
	\end{aligned}
\end{equation*}
After maximizing the latter problem, one find the input demand function:
\begin{equation}
		Y_{t}(j) = \left( \frac{P_{t}(j)}{P_{t}} \right)^{- \epsilon} Y_{t},
\end{equation}
where $P_{t}(j)$ is the price for the intermediate good $j$ and $P_{t}$ is the price for the final good.
As a reason of the zero profit condition, we get that:
\begin{equation}
		P_{t} = \left( \int_{0}^{1} P_{t}(j)^{1- \epsilon} \,dj \right)^{ \frac{1}{1- \epsilon}}
\end{equation}

\subsubsection{Intermediate-Goods Firms}
There is a continuum of monopolistic firms $j \in [0,1]$ which produce differentiated intermediate goods. Each intermediate-goods firm has access to a technology represented by the following production function:
\begin{equation}
		Y_{t}(j) = A_{t}N_{t}(j)^{1- \alpha},
\end{equation}
where $A_{t}$ represents a common level of technology, $N_{t}(j)$ is labor, and $\alpha$ is the elasticity.\\
Intermediate-goods firms take wage as given. Firms are not freely able to adjust prices. Following \citet{calvo1983staggered}, each firm may reset its price only with probability $1- \theta$ in any given period.
Let $S(t) [0,1]$ represents the set of firms not reoptimizing their posted price in period t. The aggregate price level is:
\begin{equation}
	\begin{aligned}
	P_{t} \quad & = \left[ \int_{S(t)}^{} P_{t-1}(i)^{1- \epsilon} di + (1- \theta) \left( P^{*}_{t} \right)^{1- \epsilon} \right]^{\frac{1}{1- \epsilon}}\\
		\quad & = \left[ \theta \left( P_{t-1} \right)^{1- \epsilon} + \left(1- \theta \right) \left( P^{*}_{t} \right)^{1- \epsilon} \right]^{\frac{1}{1- \epsilon}}\\
	\end{aligned}
\end{equation}
Dividing both sides by $P_{t-1}$:
\begin{equation}
	\Pi_{t}^{1- \epsilon} = \theta + (1- \theta) \left( \frac{P^{*}_{t}}{P_{t-1}} \right)^{1- \epsilon}
\end{equation}
Optimal Price Setting:
\begin{equation}
	max_{P_it}\sum_{K=0}^{\infty} \theta^{K} E_{t} \left\{ \Lambda_{t,t+K} Y_{t+K \mid t} \left[ \frac{P_{t}^{*}}{P_{t+K}}- M \frac{\psi_{t+K \mid t}}{P_{t+K}} \right] \right\} = 0
\end{equation}
The latter optimization problem leads to the following first order condition (FOC):
\begin{equation}
	\sum_{K=0}^{\infty} \theta^{K} E_{t} \left\{ \Lambda_{t,t+K} Y_{t+K \mid t} \left[ \frac{P_{t}^{*}}{P_{t+K}}- M \frac{\psi_{t+K \mid t}}{P_{t+K}} \right] \right\} = 0
\end{equation}

Log-linearization the latter FOC yields to:

\subsection{The New Keynesian Phillips Curve}
As stated above the New Keynesian Phillips Curve (NKPC) in terms of markup cap is the following:
\begin{equation} \label{NKPCmu}
	\pi_{t} = \beta E_{t} \left\{ \pi_{t+1} \right\} - \lambda \hat{\mu}_{t},
\end{equation}
where $\hat{\mu}_{t}= \mu_{t}-\mu$ is the steady state deviation of the markup.
One can rewrite the NKPC in terms of the output gap using the log-linearized expression for the average real marginal cost $mc_{t} = w_{t} - p_{t} - mpn_{t} $. This yields into the following expression, where the log deviation of real marginal cost from steady state is proportional to the log deviation of output from its flexible price counterpart:
\begin{equation}
	\hat{\mu}_{t}= \left[ \sigma + \frac{\phi + \alpha}{1- \alpha} \right] \left( y_{t} - y_{t}^{n} \right)
\end{equation}
By combining this with equation \ref{NKPCmu} yields to the following NKPC in terms of the output gap:
\begin{equation}  \label{NKPCygap}
	\pi_{t} = \beta E_{t} \left\{ \pi_{t+1} \right\} + \kappa \widetilde{y}_{t},
\end{equation}
where $\kappa \equiv \lambda \left[ \sigma + \frac{\phi + \alpha}{1- \alpha} \right] $.


\section{Estimation and Data} \label{estimation}
\subsection{Estimation}
\subsubsection{Coibion and Gorodnichenko (2015)}
OLS Regression

\subsubsection{Gali and Gertler (1999)}

\subsection{Data}
For the estimation I use quarterly US data from $1960Q1-2022Q4$.

\section{Results and Robustness Checks} \label{results}
\subsection{Results}
\begin{table}[!h]
    \input{../bld/python/tables/estimation_results.tex}
    \caption{\label{tab:python-summary}\emph{Python:} Estimation results of the
        linear Logistic regression.}
\end{table}

\subsection{Robustness Checks}


\section{Conclusion} \label{conclusion}

\setstretch{1}
\printbibliography
\setstretch{1.5}


\appendix

\begin{figure}[H]

    \centering
    \includegraphics[width=0.85\textwidth]{../bld/python/figures/MSC_Labor_share.pdf}

    \caption{\emph{Python:} Regression plot.}
    \label{fig:python-predictions}

\end{figure}

% The chngctr package is needed for the following lines.
% \counterwithin{table}{section}
% \counterwithin{figure}{section}

\end{document}
